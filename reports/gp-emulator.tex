\documentclass[12pt,a4paper]{article}
\usepackage[utf8]{inputenc}
\usepackage{amsmath}
\usepackage{amsfonts}
\usepackage{amssymb}
\usepackage{mathptmx}
\usepackage{graphicx}
\author{MK}
\title{Gaussian Process Emulator for uncertainties in sea-level rise projection}
\begin{document}

\maketitle

\section{Gaussian Process Regression}

The emulator is based on Gaussian Process Regression (GPR).
GPR is a non-parametric approach to fitting a function to a set of training data.
One major benefit of GPR is that it can estimate not only the mean but also the variance, thus allowing for uncertainty estimates of the predicitons (i.e., a Bayesian approach).
 

\begin{figure}[hbtp]
\centering
\includegraphics[scale=1]{figures/gp_constrain_slr.png}
\includegraphics[scale=1]{figures/gp_constrain_parameter.png}
\caption{Sea-level rise estimates from the emulator}
\end{figure}


\end{document}